\subsection{Purpose}
The purpose of the Students \& Companies (S\&C) project is to create a platform that efficiently matches university students with suitable internship opportunities based on their skills, experiences, and prefer- ences, while also meeting the needs of companies seeking to recruit talent. The platform streamlines the entire internship process, from profile creation and internship listings to matching, interview man- agement, and feedback collection. It aims to improve the overall internship experience, enhance the selection process, and provide universities with tools to monitor and ensure the quality of internships, ultimately benefiting students, companies, and academic institutions alike.
The S\&C platform creates a dynamic ecosystem where students, companies, and universities collaborate seamlessly to improve the internship matching process. The system’s features not only save time for both students and companies but also enhance the quality and success of internships, ultimately leading to more fulfilling and productive experiences for all parties involved.

\subsubsection{Goals}
\begin{enumerate}[label=G{\arabic*}]
\item
Allows registered students to create their profile and CV and search for internships and apply to them.
\item
Allows registered companies to create and post advertisement for available internship roles and select suitable candidate.
\item 
Provide an efficient and intelligent matchmaking process between students and companies based on the students' skills and preferences, the companies' internship descriptions.
\item
Support both students and companies during the selection process, including interview scheduling, questionnaire, feedback collection, and facilitating communication.
\item
Allows registered universities to monitor the internship statuses of their students and take action on any issues or complaints.

\end{enumerate}
\subsubsection{Scope}
The scope of this Design Document (DD) is to evaluate the technical, operational, and organizational aspects of the software engineering project in order to assess its alignment with business goals. This document discusses the architecture used in this project, highlights all the main components to be created and how they interact. This document also introduces the implementation and testing principles to be used to ensure robustness of the software.

\subsection{Definitions, Acronyms, Abbreviations}
In this section some information about terminology is provided, in order to clarify terms,
acronyms, and abbreviations used in the document, ensuring easy understanding and reference
for readers.

\subsubsection{Definitions}
\begin{itemize}
    \item \textbf{Student: } Student is a type of user who is looking for an internship.
    \item \textbf{Company: } Company is a type of user which has internship positions available and searching for students to fill those positions.
    \item \textbf{University: } University is a type of user who monitors the ongoing selection process between a student and a company.
\end{itemize}
\subsubsection{Acronyms}
\begin{itemize}
    \item \textbf{C\&S:} Company \& Students, that is the name of the platform.
    \item \textbf{UI:} User Interface
    \item \textbf{DB:} Database

\end{itemize}
\subsubsection{Abbreviations}
\begin{itemize}
\item \textbf{WPn:} n-th World Phenomena
\item \textbf{SPn:} n-th Shared Phenomena
\item \textbf{Gn: } n-th Goal
\item \textbf{Dn: } n-th Domain Assumption
\item \textbf{Rn: } n-th Requirement
\end{itemize}

\subsection{Revision History}
\begin{center}
\begin{tabular}{ | c | c | c | }
 \hline
 Revised On & Version & Description \\
 \hline
 07-Jan-2025 & 1.0 & Initial Release of Document \\
 \hline
\end{tabular}
\captionof{table}{Revision History}
\end{center}

\subsection{Reference Documents}
\begin{itemize}
    \item Assignment document A.Y. 2024/2025
\newline (Requirement Engineering and Design Project: goal, schedule and rules)
\item Software Engineering 2 A.Y. 2024/2025 Slides
\newline (Lecture slides provided during the course)
\end{itemize}

\subsection{Document Structure}
\begin{itemize}
\item \textbf{Introduction:} Introduction chapter gives a brief description of the project. This chapter discusses the different goals and scope of the S\&C system along with definitions.
\item \textbf{Architectural Design:} This section discusses the design and interaction of the different components of the S\&C system at different levels and the architecture used for the system.
\item \textbf{User Interface Design: } A set of user interfaces are mentioned is presented in this section. These interfaces are extended from those of the RSAD document.
\item \textbf{Requirements Traceability: } Each requirement from the RASD document is mapped with the design components introduced in this document.
\item \textbf{Implementation, Integration and Test Plan: } This section provides a description of the implementation integration and testing details.
\item \textbf{Effort Spent: } This section shows the effort spent to write this document by each team member.
\item \textbf{Bibliography:} This contains the references to any documents and software used to write this document.
\end{itemize}