\subsection{Purpose}
The purpose of the Students \& Companies (S\&C) project is to create a platform that efficiently matches university students with suitable internship opportunities based on their skills, experiences, and preferences, while also meeting the needs of companies seeking to recruit talent. The platform streamlines the entire internship process, from profile creation and internship listings to matching, interview management, and feedback collection. It aims to improve the overall internship experience, enhance the selection process, and provide universities with tools to monitor and ensure the quality of internships, ultimately benefiting students, companies, and academic institutions alike.
\newline The S\&C platform creates a dynamic ecosystem where students, companies, and universities collaborate seamlessly to improve the internship matching process. The system's features not only save time for both students and companies but also enhance the quality and success of internships, ultimately leading to more fulfilling and productive experiences for all parties involved.
\subsubsection{Goals}
\begin{enumerate}[label=G{\arabic*}]
\item
Allows registered students to create their profile and CV and search for internships and apply to them.
\item
Allows registered companies to create and post advertisement for available internship roles and select suitable candidate.
\item 
Provide an efficient and intelligent matchmaking process between students and companies based on the students' skills and preferences, the companies' internship descriptions.
\item
Support both students and companies during the selection process, including interview scheduling, questionnaire, feedback collection, and facilitating communication.
\item
Allows registered universities to monitor the internship statuses of their students and take action on any issues or complaints.

\end{enumerate}

\subsection{Scope}
The scope of the Students \& Companies (S\&C) platform encompasses the entire process of internship matching and management, addressing the needs of students, companies, and universities. The platform facilitates efficient connections between students seeking internships and companies that offer opportunities. This section analyzes the world phenomenon, machine phenomenon and the shared phenomena that the system will address.

\subsubsection{World Phenomena}
\begin{enumerate}[label=WP{\arabic*}]
\item 
Students seek internships to gain professional experience and improve their employability.
\item 
Companies offer internships to attract new talent and provide training.
\item 
The company gets ready for the interview and gathers questions and criteria to evaluate the student based on them.
\item
The student and the company recruiter participate in the interview.
\item
The company decides on the result of the interview and selection process.
\item
The university decides to take different actions regarding the resolution of the complaint. 
\item
Universities monitor internships to ensure compliance with educational goals and prevent issues such as internships that do not meet academic standards or are exploitative.
\end{enumerate}

\subsubsection{Machine Phenomena}
\begin{enumerate}[label=MP{\arabic*}]
\item
The system applies matching algorithms based on the analysis of students' CVs, skills, experiences, and the nature of internships (project description, company benefits).
\item
The system uses keyword searches, statistical analysis, and machine learning to recommend relevant internships to students.
\item 
The feedback received from users on previous matches is used to improve future recommendations.
\end{enumerate}

\subsubsection{Shared Phenomena}
\paragraph{World Controlled}
\begin{enumerate}[label=SP{\arabic*}]
\item
The user (Student, Company, University) signs up to the S\&C platform.
\item
The students create their CV by filling out some fields regarding their experiences, technical skills, and soft skills.
\item
The companies write about their internship and projects in the system by filling in some fields.
\item
The user (student or company) accepts or rejects the recommended match.
\item
The company creates a questionnaire for the selection process within the system.
\item
The student answers the questionnaire within the system and submits their answers.
\item
The user (student or company) inserts their available times in the calendar within the system.
\item
After the interview, the company can update the finalize decision regarding selection of a student in the S\&C system.
\item
The user (student or company) gives feedback on the recommendation of the system by giving a number to it from 1 to 5 and answering specific questions on what went wrong or could have been better.
\item
The user (student or company) clicks on the complaints button and there writes their complaint. 
\item 
The university sees the complaint about which was notified. 
\item
The university sends a message to either one of the parties that they have something to say about the complaint.
\end{enumerate}

\paragraph{Machine Controlled}
\begin{enumerate}[label=SP{\arabic*}]
\item
The system gives suggestions to students on how to write a better CV.
\item
The system gives suggestions to companies on how to write a better description of the internship.
\item
The system notifies students and companies of the best matches made among internships/projects from the statistical analyses or keyword searching.
\item
After both parties accept the recommendation, the system opens a messaging channel.
\item
The system initiates the selection process and by that sends the company’s questionnaire to the student.
\item
The system provides a calendar to both the company and student to select respective available time.
\item
After both parties insert their available times, the system suggests a time which is appropriate for both parties. 
\item
If there is no match made between the available times of company and the student, the system will notify both parties to consider other parties available times and try to work it out. 
\item
The system supports video-conferencing and creates a meeting link based on the decided date and time.
\item
The system notifies both parties about the meeting link.
\item
The system sends reminders about the interview meeting to both parties.
\item
The system asks both parties about how they evaluate the recommendation of the system and asks them to rate it from 1 to 5 and give suggestions on what could have been improved.
\item 
The system notifies the university of the student about the complaint and its details.

\end{enumerate}

\subsection{Definitions, Acronyms, Abbreviations}
In this section some information about terminology is provided, in order to clarify terms,
acronyms, and abbreviations used in the document, ensuring easy understanding and reference
for readers.

\subsubsection{Definitions}
\begin{itemize}
    \item \textbf{Student: } Student is a type of user who is looking for an internship.
    \item \textbf{Company: } Company is a type of user which has internship positions available and searching for students to fill those positions.
    \item \textbf{University: } University is a type of user who monitors the ongoing selection process between a student and a company.
\end{itemize}
\subsubsection{Acronyms}
\begin{itemize}
    \item \textbf{C\&S:} Company \& Students, that is the name of the platform.

\end{itemize}
\subsubsection{Abbreviations}
\begin{itemize}
\item \textbf{WPn:} n-th World Phenomena
\item \textbf{SPn:} n-th Shared Phenomena
\item \textbf{Gn: } n-th Goal
\item \textbf{Dn: } n-th Domain Assumption
\item \textbf{Rn: } n-th Requirement
\end{itemize}

\subsection{Revision History}
\begin{center}
\begin{tabular}{ | c | c | c | }
 \hline
 Revised On & Version & Description \\
 \hline
 22-Dec-2024 & 1.0 & Initial Release of Document \\
 \hline
\end{tabular}
\captionof{table}{Revision History}
\end{center}

\subsection{Reference Documents}
\begin{itemize}
    \item Assignment document A.Y. 2024/2025
\newline (Requirement Engineering and Design Project: goal, schedule and rules)
\item Software Engineering 2 A.Y. 2024/2025 Slides
\newline (Lecture slides provided during the course)
\end{itemize}

%what you write here is a comment that is not shown in the final text